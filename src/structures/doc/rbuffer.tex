\documentclass[10pt]{article}% ===> this file was generated automatically by noweave --- better not edit it
\usepackage{noweb}
\noweboptions{smallcode,longchunks}
\usepackage[T1]{fontenc}
\usepackage[utf8]{inputenc}
% \usepackage[french]{babel}

\author{Bernard Tatin}
\date{2013/2017}
\title{rbuffer.h, un buffer tournant}


\begin{document}
\pagestyle{noweb}
\maketitle
\tableofcontents
\section{rbuffer}

C'est un {\Tt{}buffer\ tournant\nwendquote} le plus simple possible, capable de gérer des lignes
délimitées par \emph{LF} (\texttt{'$\backslash$n'}) mais \emph{CR} (\texttt{'$\backslash$r'}) n'est pas pris en compte.

\subsection{premières définitions}
Pour limiter les calculs, le code..., la taille du buffer est une puissance de 2
d'où la définition du nombre de bits qui ouvre le bal :

\nwfilename{../rbuffer.nw}\nwbegincode{1}\sublabel{NW3ofwFz-2Sh6y8-1}\nwmargintag{{\nwtagstyle{}\subpageref{NW3ofwFz-2Sh6y8-1}}}\moddef{intro-bits~{\nwtagstyle{}\subpageref{NW3ofwFz-2Sh6y8-1}}}\endmoddef\nwstartdeflinemarkup\nwusesondefline{\\{NW3ofwFz-1p0Y9w-1}}\nwenddeflinemarkup
#define \nwlinkedidentc{_RBUFFER_BITS}{NW3ofwFz-2Sh6y8-1}   8
#define \nwlinkedidentc{RBUFFER_SIZE}{NW3ofwFz-2Sh6y8-1}    (1 << \nwlinkedidentc{_RBUFFER_BITS}{NW3ofwFz-2Sh6y8-1})
#define \nwlinkedidentc{RBUFFER_MASK}{NW3ofwFz-2Sh6y8-1}    (\nwlinkedidentc{RBUFFER_SIZE}{NW3ofwFz-2Sh6y8-1} - 1)
\nwindexdefn{\nwixident{{\_}RBUFFER{\_}BITS}}{:unRBUFFER:unBITS}{NW3ofwFz-2Sh6y8-1}\nwindexdefn{\nwixident{RBUFFER{\_}SIZE}}{RBUFFER:unSIZE}{NW3ofwFz-2Sh6y8-1}\nwindexdefn{\nwixident{RBUFFER{\_}MASK}}{RBUFFER:unMASK}{NW3ofwFz-2Sh6y8-1}\eatline
\nwused{\\{NW3ofwFz-1p0Y9w-1}}\nwidentdefs{\\{{\nwixident{{\_}RBUFFER{\_}BITS}}{:unRBUFFER:unBITS}}\\{{\nwixident{RBUFFER{\_}MASK}}{RBUFFER:unMASK}}\\{{\nwixident{RBUFFER{\_}SIZE}}{RBUFFER:unSIZE}}}\nwendcode{}\nwbegindocs{2}\nwdocspar
\nwenddocs{}\nwbegindocs{3}\nwdocspar
Il reste encore du travail...

\nwenddocs{}\nwbegincode{4}\sublabel{NW3ofwFz-1p0Y9w-1}\nwmargintag{{\nwtagstyle{}\subpageref{NW3ofwFz-1p0Y9w-1}}}\moddef{*~{\nwtagstyle{}\subpageref{NW3ofwFz-1p0Y9w-1}}}\endmoddef\nwstartdeflinemarkup\nwenddeflinemarkup
\LA{}intro-bits~{\nwtagstyle{}\subpageref{NW3ofwFz-2Sh6y8-1}}\RA{}
/**
 * @struct TSrbuffer
 * La structure gérant le buffer tournant.
 */
typedef struct \{
    volatile int in; /**< index du caractère à ajouter */
    volatile int out; /**< index du caractère à sortir */
    volatile int line_count; /**< nombre de lignes contenues dans le tampon */
    volatile char buffer[\nwlinkedidentc{RBUFFER_SIZE}{NW3ofwFz-2Sh6y8-1}]; /**< le tampon */
\} TSrbuffer;

\nwnotused{*}\nwidentuses{\\{{\nwixident{RBUFFER{\_}SIZE}}{RBUFFER:unSIZE}}}\nwindexuse{\nwixident{RBUFFER{\_}SIZE}}{RBUFFER:unSIZE}{NW3ofwFz-1p0Y9w-1}\nwendcode{}

\nwixlogsorted{c}{{*}{NW3ofwFz-1p0Y9w-1}{\nwixd{NW3ofwFz-1p0Y9w-1}}}%
\nwixlogsorted{c}{{intro-bits}{NW3ofwFz-2Sh6y8-1}{\nwixd{NW3ofwFz-2Sh6y8-1}\nwixu{NW3ofwFz-1p0Y9w-1}}}%
\nwixlogsorted{i}{{\nwixident{{\_}RBUFFER{\_}BITS}}{:unRBUFFER:unBITS}}%
\nwixlogsorted{i}{{\nwixident{RBUFFER{\_}MASK}}{RBUFFER:unMASK}}%
\nwixlogsorted{i}{{\nwixident{RBUFFER{\_}SIZE}}{RBUFFER:unSIZE}}%
\nwbegindocs{5}\nwdocspar
\paragraph{Defined Chunks}\par\noindent

\nowebchunks

\paragraph{Index}\par\noindent

\nowebindex

\end{document}
\nwenddocs{}
